\chapter{Desenvolvimento Teórico}

\section{Astrofotografia}

A astrofotografia é um ramo da astronomia e da fotografia que combina toda a ciência envolvida na documentação e registro de estrelas, constelações, planetas, meteoros, etc., com a arte da fotografia. Dentro da astrofotografia, existem variantes como planetária, solar e céu profundo \cite{livro:astropratica}. Além disso, existem diferenças entre a astrofotografia praticada profissionalmente por cientistas, em grandes telescópios, da praticada por amadores. Porém, ambas as atividades são importantes e se complementam.

As fotografias capturadas por telescópios profissionais possuem vantagens no fato como uma grande ampliação, além de foco e definição, devido aos grandes espelhos que compõe suas montagens. Contudo, isso se torna um problema para a captura de imagens mais amplas. Essas fotografias são registradas, em sua maioria, por astrofotógrafos amadores \cite{livro:astropratica}.

Além disso, a astrofotografia amadora também precisa de equipamentos que, no Brasil, custam um preço que acaba afastando uma boa parcela da população dessa prática.

\subsection{Equipamentos}

Além de uma câmera e uma lente, existem alguns equipamentos periféricos que são fundamentais para a prática da astrofotografia: tripé e disparador remoto (intervalômetro) para a câmera. O tripé é responsável por manter a câmera estável durante o registro das estrelas; o Disparador tem a função de operar a câmera remotamente para evitar que haja o operador faça a câmera tremer ao apertar algum botão e/ou também permitir a utilização do modo \textit{Bulb} das DSLR. O modo \textit{Bulb} consiste em permitir um controle total do tempo de exposição pelo operador \cite{book:bbcsky}.

\subsubsection{Câmeras}

As câmeras digitais possuem sensores CMOS (\textit{Complementary Metal-Oxide Semi-conductor}) de imagem que substituem o filme das máquinas fotográficas mais antigas. O sensor CMOS pode ter diversos tamanhos físicos diferentes e uma densidade de \textit{pixels} por polegada (dpi) diversa entre os modelos. Por norma, quanto maior for o sensor físico, mais qualidade terá a imagem final \cite{man:vanessacameras}.

Existem diversos modelos de câmeras para fotografia: compacta, super-zoom, \textit{mirrorless}, DLSR e por fim as câmeras em celulares. As câmeras \textit{mirrorless} atualmente são equiparáveis às DSLR e ambas são as mais usadas para astrofotografia, possibilitam trocar lentes e podem ser utilizadas para capturar de céu noturno. Além disso, possibilitam uma série de configurações em modo manual que não são possíveis em câmeras semi-profissionais, compactas ou de super-zoom \cite{book:bbcsky}.

\subsubsection{Lentes}
A lente é um equipamento acoplado no corpo da câmera que é responsável por focalizar a luz que invade o sensor. As lentes podem ser rígidas no corpo da câmera, no caso de modelos semi-profissionais e compactos; ou podem ser removíveis para o caso de modelos profissionais. Nesse último caso, as lentes removíveis são itens que podem ser obtidos por escolha do fotógrafo e existe uma variedade de modelos.

Esses modelos podem ser lentes fixas ou \textit{zoom}. O primeiro é um modelo de lente que possui uma distância focal fixa. Já as lentes \textit{zoom} permitem uma variação na distância focal, o que acaba gerando o \textit{zoom} óptico \cite{man:claudia7licoes}. 

\paragraph{Distância Focal}

A distância focal de uma lente é um fator medido em milímetros e é o  que determina seu ângulo de visão. Quanto maior ele for, mais fechado será o ângulo, gerando um zoom. Do contrário, quanto menor for a distância focal, maior será o ângulo de visão e consequentemente menor será o zoom da lente. A figura \ref{fig:focaldistance} ilustra essa relação da distância focal \cite{man:claudia7licoes}.

\begin{figure}[htb]
	\centering
	\caption{Efeito de zoom gerado pela variação da distância focal}
	\includegraphics[width=0.7\linewidth]{figuras/claudia-distanciafocal}
	\label{fig:focaldistance}
	\fonte{\cite{man:claudia7licoes}}
\end{figure}

Em um contexto de astrofotografia, lentes mais abertas são úteis para capturar a Via-Láctea (Figura \ref{fig:vialactea4mmSony}). Para fotografias de constelações, nebulosas e planetas distantes da Terra, é necessário uma lente mais fechada, que possibilite o enquadramento com o zoom necessário, conforme o tamanho do astro a ser fotografado. (Figura \ref{fig:jupiterSony})

\begin{figure}[!htb]
	\centering
	\caption{Fotografia da Via Láctea com lente \textit{zoom} em configuração de 4mm.}
	\includegraphics[width=0.7\linewidth]{figuras/vialactea4mm}
	\label{fig:vialactea4mmSony}
	\fonte{Autor}
\end{figure}

\begin{figure}[!htb]
	\centering
	\caption{Fotografia de Júpiter e as Luas de Galileu com lente \textit{zoom} em configuração de 205mm.}
	\includegraphics[width=0.2\linewidth]{figuras/jupiter205mm_Luas}
	\label{fig:jupiterSony}
	\fonte{Autor}
\end{figure}

\subsection{Exposição}

A exposição de uma imagem se refere à quantidade de luz captada pelo sensor da câmera. Uma imagem muito clara é considerada superexposta, um caso onde o sensor recebeu muita luz. Ao contrário, uma imagem subexposta é uma fotografia escura que recebeu pouca luz. Existem 3 parâmetros configuráveis em uma câmera profissional que são determinantes para a exposição e também para a qualidade da foto final \cite{site:eduardoemonica}. De forma geral, conseguir a exposição ideal é o principal desafio da astrofotografia de céu profundo \cite{livro:astropratica}.


\subsubsection{Tempo de Exposição}

Para captar uma imagem, a câmera possui um dispositivo que permite a passagem de luz em direção ao sensor interno que capta a imagem. Uma fotografia de longa exposição significa que a câmera permaneceu captando luz por um longo intervalo de tempo \cite{book:bbcsky}. Porém, não é possível abusar de longas exposições em alguns casos, pois a imagem pode sair "borrada" (Figura \ref{fig:velocidade}). uma pessoa correndo precisa ser fotografada em uma fração de segundo e uma paisagem, ao contrário, pode ser capturada durante mais de um segundo se a câmera estiver imóvel em um tripé.  

\begin{figure}[!htb]
	\centering
	\caption{Impacto do tempo de exposição de captura para objetos em movimento}
	\includegraphics[width=0.7\linewidth]{figuras/velocidade}
	\label{fig:velocidade}
	\fonte{Adaptado de \cite{site:eduardoemonica}}
\end{figure}

\subsubsection{Abertura}

Esse é o diâmetro do diafragme da lente, que permite a passagem de luz para o sensor (Figura \ref{fig:abertura}). Isso determina um valor "f/número". Um baixo "f/número" como f/1.8, indica um alto valor de abertura e significa dizer que a câmera irá receber mais luz \cite{book:bbcsky}. A abertura também impacta na profundidade de campo (Figura \ref{fig:profundidade}), o que significa que um valor baixo também apresenta o ônus da dificuldade focalizar.

% TODO checar questão do alto numero ou baixo número

\begin{figure}[!htb]
	\centering
	\caption{Variações de abertura de uma lente}
	\includegraphics[width=0.7\linewidth]{figuras/abertura}
	\label{fig:abertura}
	\fonte{Adaptado de \cite{site:eduardoemonica}}
\end{figure}

\begin{figure}[h]
	\centering
	\caption{Impacto da abertura na profundidade de campo}
	\includegraphics[width=0.7\linewidth]{figuras/profundidade}
	\label{fig:profundidade}
	\fonte{Adaptado de \cite{site:eduardoemonica}}
\end{figure}

\subsubsection{Sensibilidade (ISO)}

O ISO é um padrão internacional para a sensibilidade do sensor das câmeras. Essa sensibilidade também é configurável no sistema da câmera no momento da fotografia. Um valor baixo de ISO significa que o sensor precisa de mais tempo de exposição para captar mais luz, ao mesmo tempo, que reduz o ruído na imagem. (Figura \ref{fig:iso})
Um valor de ISO alto implica que a imagem final terá muito ruído, mas possibilita que ela seja registrada com um baixo tempo de exposição \cite{book:bbcsky}. O ruído agregado pelo ISO também acaba prejudicando a fotografia, reduzindo o contraste e a saturação das imagens, o que também pode levar a posterização, que é o comprometimento total da fotografia, pois a foto perde resolução e criam-se falhas nos \textit{pixels} da imagem.


\begin{figure}[!htb]
	\centering
	\caption{Variações do ISO e o ruído agregado}
	\includegraphics[width=0.7\linewidth]{figuras/ISO}
	\label{fig:iso}
	\fonte{Adaptado de \cite{site:eduardoemonica}}
\end{figure}

\subsection{Formatos de Arquivos}

As câmeras profissionais possibilitam salvar as imagens em diferentes formatos de arquivos, que inclui formato RAW, JPG ou ambos. Os arquivos JPG são uma versão reduzida dos formatos RAW, onde se aplica um algoritmo de compressão de imagens que acaba gerando perca de informações. Desse modo, arquivos RAW possuem a informação completa do sensor, sem nenhum tipo de compactação e acabam sendo muito grandes, mas permitem uma pós-produção mais precisa que acaba resultando em uma imagem com mais qualidade e detalhes \cite{book:bbcsky}.

\subsection{Rastro de Estrelas}

O movimento de rotação da terra gera um movimento aparente no céu. Ao realizar uma fotografia de longa exposição, esse movimento será visível criando o efeito de rastro de estrelas ou \textit{star trail}. (Figura \ref{fig:startrail_example})

\begin{figure}[!htb]
	\centering
	\caption{Fotografia com a captura de um \textit{star trail}}
	\includegraphics[width=0.7\linewidth]{figuras/startrail_example}
	\label{fig:startrail_example}
	\fonte{Autor}
\end{figure}

\subsubsection{Tempo de Exposição Máximo}
\label{sec:TempoMax}

Existe um limite de tempo máximo para que uma câmera fixa permaneça capturando luz sem que ocorra o fenômeno de \textit{star trail}. Esse tempo máximo depende de vários fatores, mas os dois principais são a distância focal da lente e a posição da estrela.

A distância focal é importante pois uma lente com um longo comprimento amplia a imagem, da mesma forma que amplia o rastro das estrelas. Do contrário, lentes com ângulo mais aberto, de menor comprimento focal, fazem tudo parecer pequeno, incluindo o movimento das estrelas, e isso permite um tempo de exposição maior \cite{book:astrophotographyAmateur}.

A distância de um astro até a linha do equador celestial é chamada de declinação estrelar, sendo medida em graus. Quanto menor for essa distância, mais rápido a estrela aparenta se movimentar no céu. Seja a declinação simbolizada por $\sigma$, e a distância focal nomeado $F$, em milímetros , uma equação capaz de aproximar o limite do tempo de exposição é dada em (\ref{eq:timeexp}) \cite{book:astrophotographyAmateur}. Existem ainda outras metodologias que buscam aproximar o tempo máximo de exposição. 

\begin{equation}
	t_{max} = \dfrac{343}{F\cos(\sigma)}~~[s]
	\label{eq:timeexp}
\end{equation}



\paragraph{Regra dos 500}

A regra dos 500 é uma fórmula que se baseia apenas distância focal, e o cálculo do tempo máximo é dado pela equação \ref{eq:timeexp500}. É uma regra muito simples, mas que permite uma aproximação razoável sobre o tempo máximo de exposição. Existem variantes dessa regra que alteram a constante no numerador, como a regra dos 300 ou a regra dos 400 \cite{site:500xNPF}.

\begin{equation}
	t_{max} = \dfrac{500}{F}~~[s]
	\label{eq:timeexp500}
\end{equation}

\paragraph{Regra NPF}

A regra NPF é uma evolução da equação (\ref{eq:timeexp}), que considera múltiplos fatores para recalcular a constante do numerador
\cite{site:500xNPF}. O tempo de exposição máximo é calculado pela equação \ref{eq:npf} com base na abertura da lente ($ N $), na distância focal ($ F $), no tamanho em micrômetros do sensor da câmera ($ p $), na declinação da estrela para onde a câmera será apontada ($\sigma$), e um fator de multiplicação ($ k $). O fator $ k $ é mantido em 1, porém pode ser aumentado até 3 para obter imagens mais nítidas e contrastantes \cite{site:500xNPF}. 

\begin{equation}
	t_{max} = k \cdot \dfrac{16,9 N  + 0,1 F + 13,7 p}{F\cos(\sigma)}~~[s]
	\label{eq:npf}
\end{equation}

\subsection{Empilhamento de Fotos}

O fenômeno de \textit{star trail} gera a necessidade do uso de ferramentas para compensar o movimento da Terra e permitir uma fotografia de longa exposição sem que se crie rastro. Essa compensação pode ser feita por \textit{software}, realizando-se o empilhamento de fotos de curta exposição \cite{livro:astropratica}.

O empilhamento consiste na junção de múltiplas imagens capturadas com a câmera montada em um tripé ou em uma montagem motorizada, que possibilita o somatório da luz capturada com essas fotos. Esse método de processamento é relevante para qualquer astrofotografia e possibilita a redução de ruído usando imagens de calibração \cite{book:bbcsky}. Existem inúmeros programas capazes de realizar esse processo como \textit{Deep Sky Stacker}, \textit{Sequator}, entre outros.

A combinação das imagens no pós-processamento não gera uma imagem mais luminosa ou colorida, o objetivo da combinação é o aumento da Relação Sinal Ruído (SNR). A única forma de gerar uma imagem final com mais luz e cores é realizando uma sequência de fotografias com maior tempo de exposição \cite{man:deepskystackerBetterImages}. As figuras \ref{noCalibration} e \ref{withCalibration} comparam o resultado de uma imagem que passou pelo processo de empilhamento. 

% TODO CHECAR PARENTES: 2.9(a) e posicionamento da legenda

\begin{figure}[!htb]
	\centering
	\caption{Efeito da combinação de imagens.}
	\begin{subfigure}[b]{0.49\textwidth}
		\centering
		\caption{Imagem original}
		\includegraphics[width=\textwidth]{figuras/Stack_1}
		\label{noCalibration}
	\end{subfigure}
 	\hfill
 	\begin{subfigure}[b]{0.49\textwidth}
 		\centering
	 	\caption{Empilhamento de 32 imagens}
	 	\includegraphics[width=\textwidth]{figuras/Stack_32}
	 	\label{withCalibration}
	 \end{subfigure}

	\fonte{\cite{man:deepskystackerBetterImages}}
\end{figure}


\subsubsection{Imagens de Calibração}

As fotografias registradas sobre um alvo celeste são chamadas de \textit{Light Frames} e estas podem ser empilhadas como escrito anteriormente. No entanto, é possível realizar um processo de calibração do empilhamento, fornecendo imagens de calibração \cite{man:deepskystackerfaq}.
O processo é feito combinando fotos chamadas de \textit{Dark Frames},\textit{ Bias Frames}, \textit{Flat Frames} e \textit{Dark Flat Frames} (não muito utilizado). Essas imagens são extras e precisam ser fotografadas com a câmera em condições específicas e posteriormente adicionadas ao \textit{software} durante o processo de empilhamento 
\cite{man:deepskystackerBetterImages}. O resultado entregue pelo \textit{software} será uma imagem final calibrada, como demonstra o diagrama da Figura \ref{fig:calibrationDeepSkyStacker}.

% todo refazer imagem

\begin{figure}[!htb]
	\centering
	\caption{Diagrama do processo de calibração do empilhamento sem o uso de \textit{Dark Flat Frames}}
	\includegraphics[width=0.7\linewidth]{figuras/Calibration_Alternate1}
	\label{fig:calibrationDeepSkyStacker}
	\fonte{\cite{man:deepskystackerBetterImages}}
\end{figure}


\paragraph{\textit{Dark Frames}}

Os \textit{Dark Frames} são fotografias que indicam ao software a localização do sinal de ruído das fotografias. São necessárias de 10 a 20 fotos com a lente tampada para criar a calibração, as quais devem necessariamente ser fotografadas com ISO, tempo de exposição e condições ambientais iguais aos \textit{light frames}.\cite{man:deepskystackerfaq}

\paragraph{\textit{Bias (Offset) Frames}}

Os \textit{Bias/Offset Frames} são usados para remover sinais de ruído na leitura do sensor da câmera. Essas fotografias devem ser capturadas no menor tempo de exposição possível, com lente tampada, na mesma configuração de ISO dos \textit{Light Frames}. São necessárias cerca de 10 a 20 fotos para que a calibração funcione adequadamente. A temperatura da câmera não é um fator relevante nesse caso \cite{man:deepskystackerfaq}.


\paragraph{\textit{Flat Frames}}
\textit{Flat Frames} são imagens de calibração capturadas com mesmo ISO e abertura dos \textit{light frames} colocando uma folha branca na frente da lente, incidindo luz na folha. Elas têm o objetivo de indicar a vinheta da lente(escurecimento nas bordas da imagem), além da distribuição não uniforme de luz provocada por pó ou riscos na lente. Novamente, são necessários de 10 a 20 imagens
\cite{man:deepskystackerfaq}.


\subsection{Métodos de Rastreamento}

Tendo em vista o limite do tempo de exposição e o movimento de rotação da Terra, discutidos na seção \ref{sec:TempoMax}, os \textit{softwares} de empilhamento possuem algoritmos que compensam a rotação das estrelas, rotacionando as imagens fotografadas no sentido oposto, e realizando o empilhamento dessas imagens após esse ajuste das fotos. Esse método compensa o ruído, mas como os tempos de exposições são curtos, torna-se mais difícil obter cor e contrastes nos objetos celestes. Isso só é possível capturar aumentando o tempo de exposição. 

Então, para realizar astrofotografias de longa exposição, é necessário o uso de um rastreador físico que movimenta a câmera no sentido de rotação aparente das estrelas, garantindo que não ocorrerá o efeito de \textit{star trail}. Existem dois métodos de rastreamento: Alt-Azimutal e Equatorial \cite{book:bbcsky}.

\subsubsection{Alt-Azimutal}

Uma montagem Alt-Azimutal funciona movendo uma câmera ou um telescópio por meio dos eixos vertical e horizontal, alterando o azimute e a altitude simultaneamente. Isso requer um sistema com dois motores para realizar o rastreamento, o que torna essa montagem mais cara e complexa. Além disso, para astrofotografias, essa montagem acaba não sendo indicada pois ela não consegue compensar a rotação aparente dos astros, que também é gerado pelo movimento de rotação da terra (Figura \ref{fig:altazimuterotation}) \cite{book:bbcsky}. 

\begin{figure}[!htb]
	\centering
	\caption{O enquadramento não rotaciona com o astro na montagem Alt-Azimutal}
	\includegraphics[width=0.45\linewidth]{figuras/altazimuterotation}
	\label{fig:altazimuterotation}
	\fonte{Adaptado de \cite{book:bbcsky}}
\end{figure}

% todo adaptar essa imagem

\subsubsection{Equatorial}

Ao contrário do modelo de montagem comentado na seção anterior, uma montagem equatorial consegue compensar a rotação aparente dos astros (Figura \ref{fig:equatorialrotation}) e, por esse motivo, é a melhor opção de mecanismo para realizar astrofotografias. Isso ocorre pois essa construção realiza o movimento da câmera de forma circular, na mesma velocidade de rotação aparente da Terra, após alinhar o eixo de altitude junto com o meridiano polar (eixo norte-sul) \cite{book:bbcsky}. Esse sistema requer somente um motor, porém, precisa também de um método acurado de alinhamento com o meridiano e isso será explorado na próxima seção. 

\begin{figure}[!htb]
	\centering
	\caption{O enquadramento se mantêm constante na montagem equatorial}
	\includegraphics[width=0.45\linewidth]{figuras/equatorialrotation}
	\label{fig:equatorialrotation}
	\fonte{Adaptado de \cite{book:bbcsky}}
\end{figure}

\section{Plataformas Equatoriais}

Plataformas Equatoriais são mecanismos que se baseiam em uma montagem equatorial e podem ter os mais diversos tamanhos. Existem modelos para telescópios e outros específicos para astrofotografia, sendo este último que será o foco deste trabalho. Essa montagem requer o alinhamento do eixo de rotação da plataforma, com o eixo de rotação celeste que, para uma pessoa localizada no hemisfério norte, será o polo norte polar, e para alguém no hemisfério sul, será o polo sul polar (Figura \ref{fig:celestialchart}). 

\begin{figure}[!htb]
	\centering
	\caption{A posição do polo norte/sul celestial, no céu, depende da posição geográfica (latitude) da pessoa/equipamento de observação e é alinhado com o eixo de rotação da terra.}
	\includegraphics[width=0.8\linewidth]{figuras/celestialchart}
	\label{fig:celestialchart}
	\fonte{\cite{livro:starwatch:v1}}
\end{figure}

Depois que o eixo da plataforma está alinhada com o polo celeste, ela começa a rotacionar no sentido da rotação do planeta, e isso pode ser feito em diferentes modelos de estrutura.

\subsection[Modelos de Montagem]{Modelos de Montagem \textit{Barn Door}}
\textit{Barn Doors} são um modelo de montagem que se caracterizam por funcionar com duas bases acopladas, onde uma é fixada no tripé do fotógrafo, e a outra é móvel, fixando a câmera que será rotacionada para acompanhar o movimento aparente do céu (Figura \ref{fig:barndoorexample}). O funcionamento é igual à abertura de uma porta com dobradiças \cite{site:pentaxBarnDoor}. 

\begin{figure}[!htb]
	\centering
	\caption{Exemplo de Montagem Barn Door}
	\includegraphics[width=0.5\linewidth]{figuras/barndoorexample}
	\label{fig:barndoorexample}
	\fonte{\cite{artigo:garySeronik}}
\end{figure}


Esses modelos são tradicionalmente conhecidos pela comunidade de astrofotografia por serem de baixo custo, pois é possível automatizar o movimento da câmera sem a necessidade de um motor demasiadamente potente e caro. Além disso, comparando a modelos de montagem usado em sistemas comerciais, um sistema \textit{Barn Door} pode ser facilmente modificado ou reparado, e é, geralmente mais estável. Perdem para modelos comerciais no quesito transportabilidade e precisão \cite{site:pentaxBarnDoor}. 
 
Dentre os modelos de \textit{Barn Door}, os mais comuns são: Montagem com braço simples, com braço Duplo e montagem Curva. A primeira (Figura \ref{fig:singleArm}) é composta por uma base fixa conectada a base da câmera que é movida por um eixo perpendicular à parte fixa. Esse sistema acaba tendo algumas limitações para manter uma variação constante no ângulo de rotação da câmera, que, apesar da elevação do eixo ser constante, o ângulo de rotação não é, o que acumulará erro de rastreamento. Isso gera erros de rastreamento que limitam o uso desse modelo para exposições com, no máximo, 15 minutos \cite{artigo:davidtrottinventions}. 

\begin{figure}[!htb]
	\centering
	\caption{Barn Door com Braço Simples}
	\includegraphics[width=0.6\linewidth]{figuras/bracosimples}
	\label{fig:singleArm}
	\fonte{Adaptado de \cite{artigo:davidtrottinventions}}
\end{figure}

David Trot desenvolveu um mecanismo de Braço Duplo (Figura \ref{fig:doublearm}) que consegue aumentar para até 1h o tempo máximo de exposição, em comparação com o modelo anterior. Contudo, tem como desvantagem a complexidade do sistema, que exige várias peças em um diagrama que pode se tornar demasiadamente complexo.

% TODO: traduzir
\begin{figure}[!htb]
	\centering
	\caption{Diagrama de montagem do mecanismo de Braço Duplo}
	\includegraphics[width=0.6\linewidth]{figuras/heavy-duty-double-arm-barndoor-building-plans-3}
	\label{fig:doublearm}
	\fonte{Adaptado de \cite{artigo:davidtrottinventions}}
\end{figure}

Por fim, a Montagem Curva é composta por uma barra roscada curva, que não possui o problema de compensação de velocidade, pois ela acompanha a curvatura do movimento da plataforma. A problemática dessa montagem provém da curvatura do eixo de rotação, que pode trazer questões relacionadas a estabilidade, devido à necessidade de folga nas bases (Figura \ref{fig:montagemCurva}). Outro detalhe é que a barra não pode ser rotacionada; a movimentação deve ser feita através de uma rosca que é rotacionada na base, e realiza o deslocamento da barra roscada para cima ou para baixo \cite{site:pentaxBarnDoor}.  

 
 % TODO: traduzir
 \begin{figure}[!htb]
 	\centering
 	\caption[Modelo de Montagem Curva]{Modelo de Montagem Curva e o problema da folga no mecanismo}
 	\includegraphics[width=0.6\linewidth]{figuras/montagemCurva}
 	\label{fig:montagemCurva}
 	\fonte{Adaptado de \cite{artigo:edjonescurved}}
 \end{figure}

Comparando os atributos dos modelos de Barn Door, a montagem curva é a mais apropriada para o projeto proposto, visto que não demanda a construção de um mecanismo demasiadamente complexo. Dessa forma, ela reduz a quantidade de materiais e processos, bem como a lógica de controle do motor, minimizando ainda mais os custos, ao passo que potencializa o resultado do projeto. 

\subsection{Métodos de Alinhamento Polar}
O alinhamento com o polo norte/sul celeste é fundamental para a execução da astrofotografia através de uma plataforma equatorial, e erros podem comprometer o funcionamento do sistema. Quanto mais bem alinhada está a plataforma, maior será o tempo de exposição que ela conseguirá obter sem gerar rastros nas estrelas. Para realizar esse alinhamento, existem dois métodos: Localizar a estrela Polar que indica a posição do polo celeste, e/ou utilizar de instrumentação para posicionar e inclinar a plataforma nos valores de azimute e inclinação corretos, dada a posição da plataforma no planeta.

\subsubsection{Localização da Estrela Polar}
Existem duas formas de alinhamento por meio da localização no céu da estrela polar. A primeira delas é um \textit{laser}, que é alinhado com a estrela polar para indicar o alinhamento (Figura \ref{fig:alinhamentolaser}). No entanto, o \textit{laser} não é um método extremamente confiável, pois a estrela polar não é exatamente centrada no polo norte celeste, dessa forma o alinhamento não fica preciso. A única vantagem desse método é o custo que é, em média, na casa de 100 reais.\footnote{Considerando Modelos pesquisados em agosto de 2021}

 \begin{figure}[!htb]
	\centering
	\caption{Alinhamento por laser}
	\includegraphics[width=0.3\linewidth]{figuras/alinhamentolaser}
	\label{fig:alinhamentolaser}
	\fonte{\cite{site:testingMSM}}
\end{figure}


O segundo método envolve uma luneta que permita a identificação das constelações que indicam o polo norte/sul (Figura \ref{fig:luneta}). No hemisfério Norte, procura-se a estrela Polaris; no hemisfério Sul, busca-se a constelação de Sigma Octantis e o Cruzeiro do Sul. Ele é um método bem confiável, porém, no hemisfério sul, isso normalmente é mais difícil, pois essas constelações são de alta magnitude. Isso significa dizer que possuem um baixo brilho, tornando-as difíceis de serem localizadas no céu. As lunetas buscadoras têm um custo bem variado, mas são normalmente comercializadas fora do Brasil e tem um custo que começa na casa dos 80 dólares, ou 415 reais.\footnote{Considerando Modelos pesquisados em agosto de 2021, com câmbio de US\$ 1 = R\$5,19}

 \begin{figure}[!htb]
	\centering
	\caption{Visor de uma luneta para astrofotografia, contendo marcadores para alinhar as estrelas}
	\includegraphics[width=0.5\linewidth]{figuras/luneta}
	\label{fig:luneta}
	\fonte{\cite{site:bresserpolarscope}}
\end{figure}

Ambos os métodos possuem a desvantagem de requerer um céu limpo na região polar (Figura \ref{fig:temporuim}) e baixos níveis de poluição luminosa para identificar as estrelas, porém, apresentam a simplicidade como vantagem. O laser é o menos confiável, pois existe uma defasagem entre as constelações e o polo celeste, que só é possível de ser compensado com o uso da luneta. 

 \begin{figure}[!htb]
	\centering
	\caption{Alinhamento impossível sem colaboração do tempo limpo}
	\includegraphics[width=0.5\linewidth]{figuras/temporuim}
	\label{fig:temporuim}
	\fonte{\cite{site:nyxtechtips}}
\end{figure}


\subsubsection{Instrumentação}

Através de instrumentos de medição, é possível realizar o alinhamento da plataforma equatorial separando o processo em duas etapas. Na primeira etapa é feito o Ajuste do Azimute para alinhar a plataforma com o polo norte geográfico. Posteriormente ela deve ser inclinada até o ângulo referente ao polo norte celeste que é dado pelo ângulo da latitude do local onde a plataforma está localizada, realizando um ajuste de elevação. 

\paragraph{Ajuste de Azimute}
O ajuste do azimute pode ser realizado com uma bússola ou um magnetômetro que possibilite indicar o polo norte magnético. No entanto, o polo norte geográfico possui uma diferença com o polo norte magnético devido à oscilação do campo magnético do planeta, essa discrepância é chamada de Declinação Magnética. 

Além disso, devido à inconsistência do campo magnético, o valor da declinação é diferente para cada localização do planeta, mas pode ser calculada usando modelos magnéticos globais que são o resultado de pesquisas com medidores em satélites e apresentam valores com acurácia de 0,5 graus \cite{site:noicDecMag}.

\paragraph{Ajuste de Elevação}
O Ajuste de elevação para uma determinada latitude pode ser feito através de marcações de ângulo em um eixo dobrável (Figura \ref{fig:marcacao_latitude}), ou com sensores de posição inercial: acelerômetro e giroscópio \cite{site:driftLupus}. É evidente que o primeiro método não é preciso, pois depende obrigatoriamente de um bom processo de manufatura e calibração, além da falta de precisão para ângulos de latitude com valores decimais. 

\begin{figure}[!htb]
	\centering
	\caption{Marcação da latitude para ajuste de elevação}
	\includegraphics[width=0.45\linewidth]{figuras/marcacao_latitude}
	\label{fig:marcacao_latitude}
	\fonte{\cite{site:driftLupus}}
\end{figure}

\subsubsection{Método \textit{Drift}}

O alinhamento da plataforma com o polo celeste é fundamental para uma longa exposição usando uma lente com grande comprimento focal. Se a montagem não estiver corretamente alinhada, ela não irá conseguir impedir o surgimento de rastro na fotografia por um longo período de tempo, dessa forma, quanto melhor for o alinhamento, maior será o tempo de exposição sem a geração de rastro \cite{book:bbcsky}.  

Plataformas que são alinhadas com uma luneta polar conseguem atingir normalmente 180s de exposição. Se mais tempo for necessário ou for usada uma lente com muita ampliação, então o método \textit{Drift} de alinhamento é mandatório. Ainda que, não é possível realizar esse ajuste de precisão sem que a plataforma já tenha sido previamente alinhada pelos outros métodos de menor acurácia que já foram explicados nas seções anteriores \cite{book:bbcsky}. 

O método consiste em localizar, primeiramente, uma estrela brilhante o suficiente para visualizar no visor da câmera, que esteja na linha do equador polar. Habilitando uma linha de grade no visor, o usuário deve alinhar a estrela escolhida de forma que fique centralizada no cruzamento da grelha (Figura \ref{fig:driftgrelha1}) e, então, ligar o rastreador, observando se haverá movimentação da estrela para algum dos lados do visor. Na sequência, o usuário deve apontar para uma estrela ao leste ou oeste, no horizonte, e observar para qual lado (Esquerda/Direita ou Norte/Sul) do visor haverá fuga (Figura \ref{fig:driftgrelha2}). O lado (esquerda ou direita) que indica norte ou sul depende para onde a câmera estiver sendo apontada e a orientação dela \cite{book:bbcsky}.

\begin{figure}[!htb]
	\centering
	\caption{Estrela centralizada na grelha do visor da câmera}
	\includegraphics[width=0.7\linewidth]{figuras/driftgrelha1}
	\label{fig:driftgrelha1}
	\fonte{Adaptado de \cite{site:driftLupus}}
\end{figure}

\begin{figure}[!htb]
	\centering
	\caption{Estrela apresentando fuga para a direita, que neste caso, aponta para o sul}
	\includegraphics[width=0.7\linewidth]{figuras/driftgrelha2}
	\label{fig:driftgrelha2}
	\fonte{Adaptado de \cite{site:driftLupus}}
\end{figure}

Dependendo para qual lado do visor (norte/sul) houver fuga, deverá ser feito um ajuste específico no alinhamento, que também é diferente para cada hemisfério do planeta. O fotógrafo deve saber para qual lado do visor é norte ou sul, pois será isso que determinará corretamente a correção a ser realizada na plataforma (Tabela \ref{tab:drift}). A primeira estrela ajuda a ajustar o azimute da plataforma, a segunda colabora para uma elevação correta. Além disso, a precisão do ajuste dependerá do tempo que a estrela irá levar para sair da marcação do visor, e o processo de repete até que a estrela permaneça parada pelo tempo mínimo desejado pelo fotógrafo para a lente que estiver usando \cite{book:bbcsky}. 

\begin{table}[!htp]
	\centering
	\caption{Ajustes do Método \textit{Drift} mediante cada caso}
	\label{tab:drift}		
		\begin{tabular}{c|c|c|c}
			\makecell{Localização\\da estrela} & Lado da Fuga & \makecell{Correção\\(Hemisfério Sul)}& \makecell{Correção\\(Hemisfério Norte)}\\  \hline
			\multirow{2}{*}{ Meridiano } & Norte & Azimute para Leste & Azimute para Leste \\ \cline{2-4}
			& Sul & Azimute para Oeste & Azimute para Oeste \\ \hline
			\multirow{2}{*}{ Leste } & Norte & Altura para cima & Altura para baixo \\ \cline{2-4}
			& Sul & Altura para baixo & Altura para cima \\ \hline
			\multirow{2}{*}{ Oeste } & Norte & Altura para baixo & Altura para cima \\ \cline{2-4}
			& Sul & Altura para cima & Altura para baixo \\ 
		\end{tabular}
	
	\fonte{Adaptado de \cite{site:driftLupus}}
\end{table}

É importante ressaltar que para esse método ser possível, o tripé onde a plataforma é montada deve possuir \textit{knobs} para ajuste mais acurado, como na cabeça de tripé da Figura \ref{fig:ballhead}, que possui um \textit{knob} para movimentação somente de azimute, outro para movimento livre, e um terceiro para um ajuste mais preciso de ambas as posições.

\begin{figure}[!htb]
	\centering
	\caption{Cabeça de Tripé com possibilidade para ajuste de precisão}
	\includegraphics[width=0.35\linewidth]{figuras/ballhead}
	\label{fig:ballhead}
	\fonte{(Optison, c2021)}
\end{figure}

\subsection{Soluções Comerciais Existentes}

Existem inúmeras soluções comerciais para o problema proposto, porém, todos eles usam uma luneta ou um \textit{laser} como método de alinhamento polar. Existem produtos com diferentes especificações e orçamentos. A Tabela \ref{tabela_benchmark} ilustra alguns sistemas, dentre os disponíveis no mercado, comparando suas funcionalidades. 

% todo: mencionar o rastreador compacto moveshootmove.com

\begin{table}[htb]
	\caption{Comparativo das Soluções de Mercado}
	\begin{tabular}{l|cccc}
		& Nyx Tracker & iOptron & Vixen Optics & SkyWatcher \\ \hline
		Preço (US\$) & 115 & 299 & 399 & 299 \\\hline
		Carga Máxima (kg) & 2.25 & 3 & 2 & 3 \\\hline
		Erro periódico (arcsec) & 115 & 100 & 50 & 50 \\\hline
		Volume $ (cm^2) $ & 155 & 490 & 323 & 220 \\\hline
		Peso (kg) & 0,4 & 1,15 & 0,79 & 0,72 \\\hline
		Alinhamento & \textit{Laser} & \textit{Polar Scope} & \textit{Polar Scope} & \textit{Polar Scope} \\
	\end{tabular}
	\label{tabela_benchmark}
	\fonte{Adaptado de \cite{site:nyxtech}}
\end{table}

Contudo, na realidade brasileira, o preço mostrado passaria ainda por impostos, tornando a compra mais inviável. O Nyx Tracker (Figura \ref{fig:nyxtracker}) é o único da lista que possui uma estrutura de \textit{Barn Door}, e é o sistema mais acessível, porém o método de alinhamento é de difícil execução no hemisfério sul.

\begin{figure}[h]
	\centering
	\caption{Nyx Tracker}
	\includegraphics[width=0.3\linewidth]{figuras/nyxtracker}
	\label{fig:nyxtracker}
	\fonte{\cite{site:nyxtech}}
\end{figure}


\section{Objetivos}

Pelo \textit{benchmark} exposto, fixaram-se como objetivos o desenvolvimento de uma solução robusta, visualmente elegante, e que consiga se aproximar das propriedades do modelo comercial mais acessível, com o custo inferior a 115 dólares. Além disso, deve ter como diferencial um aplicativo que permita uma fácil interação do usuário com o sistema, facilitando o processo de configuração e alinhamento polar.

\section{Protocolos de Comunicação}

\subsection{Serial}
\subsubsection{UART}
Velocidade, Falhas de comunicação, Guidelines de Design de PCB

\subsubsection{I2C}

Endereçamento, Velocidade, Guidelines de Design de PCB

\subsection{Bluetooth}

\section{Sensores e Atuadores}

\subsection{Acelerômetro}
\subsection{Giroscópio}
\subsection{Magnetômetro}

\subsection{GPS}

\subsection{Motor de Passo}

Driver, formas de Acionamento...

\section{Microcontroladores}

\subsection{Arduino Nano}

Justificativa, diagrama do Arduíno

\section{Interface Gráfica}

\subsection{Princípios e Diretrizes}

Os princípios e as diretrizes comumente utilizados em interfaces humano-computador giram em torno dos seguintes tópicos: correspondência com as expectativas dos usuários; simplicidade nas estruturas das tarefas; equilíbrio entre controle e liberdade do usuário; consistência e padronização; promoção da eficiência do usuário; antecipação das necessidades do usuário; visibilidade e reconhecimento; conteúdo relevante e expressão adequada; e projeto para erros \cite{BarbosaEtAl2021InteracaoHumanoComputadorExperiencia}.
Esse conjunto de princípios são conhecidos como heurística de Nielsen, pois são aplicáveis em qualquer sistema, independente de casos específicos.

\subsubsection{Visibilidade dos status do sistema}

O sistema deve sempre manter o usuário atualizado sobre as condições de operação com uma taxa de atualização condizente para a informação. Ao informar o status da bateria, por exemplo, o usuário do \textit{smartphone} consegue predizer quanto tempo de uso ainda terá e irá conseguir manejar sua interação com base nessa previsibilidade \cite{site:nielsen}.

\subsubsection{Comunicar-se com o mundo real}
O Projeto tem que se comunicar com o usuário na língua do usuário. Se um brasileiro não sabe inglês, ele "ficará perdido" nos Estados Unidos. Da mesma forma, o desenvolvedor não pode assumir que o usuário entenderá o aplicativo somente pelo fato do desenvolvedor ter feito algo que ele próprio entenda. É sempre recomendado conferir a linguagem do sistema com um conjunto grande de pessoas para evitar mal entendidos.

Quando o usuário não entende a língua do sistema, ele se sente afastado e irá deixar de usar a plataforma. É interessante que a plataforma tenha \textit{designs} semelhantes com objetos do mundo real, dessa forma, o usuário se sente "contemplado" e consegue facilmente fazer a conexão entre o mundo real e a plataforma \cite{site:nielsenRealWorld}.

\subsubsection{Liberdade de Controle do Usuário}

Por vezes, a pessoa que está realizando um processo em um sistema pode cometer um engano. Esse evento pode levar a situações de erro que não devem comprometer a experiência. Por isso, os usuários precisam de uma “saída de emergência” claramente marcada para sair do estado indesejado. Isso reduz a sua ansiedade e o medo de errar, pois ele sabe que os erros podem ser contornados \cite{BarbosaEtAl2021InteracaoHumanoComputadorExperiencia}.

\subsubsection{Consistências e Padrões}

É importante que o sistema mantenha uma consistência entre suas telas, ou mesmo em grandes plataformas, ou seja, que os múltiplos programas tenham o mesmo padrão, com funções localizadas no mesmo lugar, com nomes similares e com um \textit{disign} similar. Exemplo disso são as telas dos aplicativos do Google Docs: todos possuem o mesmo estilo de menu. Idem para o Microsoft Office.

A consistência também se estende aos ícones. O ícone que representa um botão, por exemplo, é importante que seja consistente em estilo com os demais. Eles podem ser mais preenchidos, \textit{clean}, neutros ou suaves. O que importa nesse caso, é que sejam todos padronizados \cite{site:nielsenIcon}.


\subsubsection{Prevenção de erros}

Uma forma de prevenção é oferecer sugestões numa caixa de pesquisa, por exemplo. Em situações de rotina, como disparar um lembrete, a tela de criação pode oferecer uma sugestão padrão de um modelo que faça sentido para o usuário. Para evitar corrupção de dados pelo usuário durante o cadastro, é possível sugerir ao usuário o preenchimento de números de forma truncada, fazendo o pós-processamento para ler o número corretamente.

\subsubsection{Relembrar o usuário é mais fácil do que o usuário relembrar}

Quando o usuário precisa repensar sobre algo incomum na memória, ele despende muito tempo. Então, quando a plataforma exige uma lembrança do usuário para entender algo, isso limita a experiência e incorre em perda de tempo ou confusão.

Por isso, é mais interessante realizar a exigência com uma possível sugestão de resposta correta. A recognição de algo é muito mais prática para a mente humana, pois ao mostrar para o cérebro algo relacionado com o que precisa lembrar-se, dispara-se a memória de forma mais efetiva. Dar uma pista para o cérebro é mais eficiente do que simplesmente perguntar sem oferecer nada \cite{site:nielsenRecall}.

\subsubsection{Torne o sistema flexível e eficiente}

Atalhos, personalização e customização. Com esses fatores é possível melhorar a usabilidade para aqueles que não são mais novatos no \textit{software} e isso ajuda a manter esses usuários ativos. Um fotógrafo experiente, que está acostumado com os atalhos de teclado nos aplicativos da Adobe, teria muita dificuldade se o teclado viesse a falhar, pois a mente já assimilou os atalhos mais usados e eles fazem diferença na velocidade com que o profissional interage com o software \cite{site:nielsenFlexibility}.

\subsubsection{Tenha um projeto minimalista}

Um projeto é minimalista significa usar elementos simples num arranjo onde desenho e a interface combinem de forma agradável sem chamar a atenção de forma desnecessária, colaborando com que o usuário foque somente naquilo que é necessário \cite{site:nielsen}.

\subsubsection{Ajude o usuário a entender e se recuperar de erros}

O usuário precisa entender quando o sistema não está funcionando bem e como fazê-lo voltar à normalidade. As mensagens de erro devem ser expressas de uma forma simples, indicando o possível problema e a solução. 
Cores vermelhas e pretas ajudam a demonstrar o sinal de erro para o usuário \cite{site:nielsenError}.

\subsubsection{Tire dúvidas e documente o sistema}

Existem duas formas de ajudar o usuário e tirar suas dúvidas. A primeira é de forma proativa, onde a aplicação guia o usuário para se familiarizar com a interface. Outra forma é por uma seção com perguntas e respostas, a qual ajuda os usuários a se tornarem mais independentes com a aplicação, resolvendo seus próprios problemas e filtrando os casos que precisam de suporte para a equipe técnica da plataforma \cite{site:nielsenHelpandDoc}.

\subsection{Android}
\subsubsection{Ambiente de Desenvolvimento}

O \textit{Android Studio} é o ambiente de desenvolvimento integrado oficial para a criação de aplicativos \textit{Android} e é baseado no \textit{IntelliJ IDEA}. Ele oferece uma série de Recursos que possibilitam a confecção de um aplicativo: Sistema de compilação flexível baseado em \textit{Gradle}; Um emulador rápido com suporte a vários recursos; ambiente unificado que possibilita o desenvolvimento para qualquer dispositivo \textit{Android}, incluindo relógios e televisões; integração com \textit{GitHub} para \textit{backup} e documentação do código; entre outras funções que possibilitam analisar o desempenho de um aplicativo em tempo real, bem como fazer \textit{updates}. \cite{site:androidstudio}

\subsubsection{Linguagens de Programação}

Existem soluções de desenvolvimento \textit{Android} mais \textit{user-friendly} como \textit{APP Inventor} ou \textit{Kodular}, porém, essas interfaces não garantem ao desenvolvedor um pleno controle do aplicativo, e muitas vezes acabam limitando o projeto da interface. Por isso, usar linguagens de programação nativas é uma abordagem mais interessante para aplicativos mais completos. É possível criar aplicativos com diversas linguagens, mas somente duas são nativas e permitem realizar aplicações que podem usar de todo o poder de processamento de um \textit{smartphone}: Java e \textit{Kotlin}.

Em 2017, \textit{Kotlin} foi definido pela Google como sendo a principal linguagem de desenvolvimento \textit{Android}. Ela é muito mais nova que Java, sendo desenvolvida em pela JetBrains. A grande motivação de se usar \textit{Kotlin} para o desenvolvimento reside no fato de ser uma linguagem segura para prevenção de objetos nulos, operando em paralelo com qualquer código em Java e dando opções de co-rotinas. Além disso, ao comparar dois códigos com a mesma função, um escrito em Java, outro em \textit{Kotlin}, o segundo pode ser até 40\% mais compacto, o que implica em uma linguagem mais concisa e compreensível entre desenvolvedores. A desvantagem de se usar \textit{Kotlin}, para este trabalho, é somente a falta de uma comunidade grande, comparando com Java, o que limita o suporte para eventuais problemáticas de desenvolvimento \cite{site:kotlinxjava}.

Dentro do ambiente de desenvolvimento usa-se também linguagem de arquivos XML para a criação de interfaces gráficas (\textit{layouts}), bem como a escrita dos vetores, animações, e arquivos de configuração do aplicativo e temas de \textit{layout}. Para armazenamento de dados dentro do aplicativo, normalmente usa-se um banco de dados que é operado com códigos de consulta \textit{SQL}.

\subsubsection{\textit{Material Design}}
Existem uma série de diretrizes de projeto fornecidas pela Google para guiar o desenvolvimento de aplicativos \textit{Android}. Essas informações são fornecidas principalmente pela biblioteca \textit{Material Design}, que fornece pacotes facilmente implementáveis de \textit{layouts} para aplicações responsivas e padronizadas.

A biblioteca colabora com o desenvolvedor fornecendo ícones, tipografia, cores e componentes gráficos que trazem uma imersão para o usuário de forma simples e minimalista. Os \textit{designs} se inspiram no mundo real, facilitando a comunicação com o usuário \cite{site:materialdesign}.

\subsection{Usuários}

