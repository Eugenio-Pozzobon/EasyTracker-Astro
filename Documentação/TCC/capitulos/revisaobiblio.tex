\chapter{Desenvolvimento Teórico}

\section{Astrofografia}

A astrofotografia é um ramo da astronomia e da fotografia que combina toda a ciência envolvida na documentação e registro de estrelas, constelaçoes, planetas, meteoros, etc.; com a arte da fotografia. Dentro da astrofotografia, existem variantes de fotografia como planetária, solar e céu profundo. \cite{livro:astropratica}. Além disso, existem diferenças entre a astrofotografia praticada profissionalmente por cientistas, em grandes telescópios, da praticada por amadores. Porém, ambas as atividades são importantes e se complementam.

As fotografias capturadas por telescópios profissionais possuem vantagens no fato de que essas imagens conseguem uma grande ampliação, foco e definição, devido aos grandes espelhos que compõe suas montagens. Porém, isso se torna um problema para a captura de imagens mais amplas e conseguir observar outros detalhes; essas fotografias são registradas em sua maioria por astrofotografos amadores. \cite{livro:astropratica}

Além disso, a astrofotografia amadora também precisa de equipamentos que, no Brasil, custam um preço que acaba afastando uma boa parcela da população para a prática da observação celeste. 

\subsection{Equipamentos}

Além de uma câmera e uma lente, existem alguns equipamentos periféricos que são fundamentais para a prática da astrofotografia: Tripé e Disparador para a câmera. O tripé é responsável por manter a câmera estável durante o registro das estrelas; o Disparador tem a função de operar a câmera remotamente para evitar que haja o operador faça a câmera tremer ao apertar algum botão e/ou também permitir a utilização do modo \textit{Bulb} das DSLR. O modo \textit{Bulb} consiste em permitir um controle total do tempo de exposição pelo operador. \cite{book:bbcsky}

\subsubsection{Câmeras}

DSLR e Mirrorless

\subsubsection{Lentes}

\subsection{Exposição}

A exposição de uma imagem se refere a quantidade de luz captada pelo sensor da câmera. Uma imagem muito clara é uma imagem superexposta, um caso onde o sensor recebeu muita luz; ao contrário, uma imagem subexposta é uma fotografia escura que recebeu pouca luz. Existem 3 parâmetros configuraveis em uma câmera profissional que são determinantes para a quantidade de luz a ser captada pelo sensor e também para a qualidade da foto final. \cite{site:eduardoemonica} De forma geral, conseguir a exposição ideal é o principal desafio da astrofotorafia de céu profundo. \cite{livro:astropratica}


\subsubsection{Velocidade}

Para captar uma imagem, a câmera possui um dispositivo que permite a entrada de luz no sensor interno que capta a imagem. O tempo que a câmera permite a passagem de luz determina a velocidade do disparador dela. Uma fotografia de longa exposição significa que a câmera permaneceu captando luz por um longo intervalo de tempo. \cite{book:bbcsky} Porém, não é possível abusar de longas exposições em alguns casos pois a imagem pode sair "borrada" (Figura \ref{fig:velocidade}); uma pessoa correndo precisa ser fotografada em uma fração de segundo, uma paisagem, ao contrário, pode ser capturada durante mais de um segundo se a câmera estiver imóvel em um tripé.  

\begin{figure}[htb]
	\centering
	\caption{Impacto da Velocidade de captura para objetos em movimento}
	\includegraphics[width=0.7\linewidth]{figuras/velocidade}
	\label{fig:velocidade}
	\fonte{Adaptado de \cite{site:eduardoemonica}}
\end{figure}

\subsubsection{Abertura}

Esse é o diâmetro da abertura da lente, que permite a passagem de luz para o sensor (Figura \ref{fig:abertura}). Isso determina um valor "f/número". Um baixo "f/número" como f/1.8, indica um alto valor de abertura e significa dizer que a câmera irá receber mais luz. \cite{book:bbcsky} A abertura também impacta na profundidade de campo (Figura \ref{fig:profundidade}), o que significa que um valor baixo também apresenta o ônus da dificuldade de focar em objetos.

\begin{figure}[h]
	\centering
	\caption{Variações de Abertura de uma lente}
	\includegraphics[width=0.7\linewidth]{figuras/abertura}
	\label{fig:abertura}
	\fonte{Adaptado de \cite{site:eduardoemonica}}
\end{figure}

\begin{figure}[h]
	\centering
	\caption{Impacto da abertura na profundidade de campo}
	\includegraphics[width=0.7\linewidth]{figuras/profundidade}
	\label{fig:profundidade}
	\fonte{Adaptado de \cite{site:eduardoemonica}}
\end{figure}

\subsubsection{Sensibilidade (ISO)}

O ISO é um padrão internacional para a semsibilidade do sensor das câmeras. Essa sembilidade também é configurável no sistema da câmera no momento da fotografia. Um valor baixo de ISO significa que o sensor precisa de mais tempo de exposição para captar mais luz, ao mesmo tempo que reduz o ruído na imagem. (Figura \ref{fig:iso})
Um valor de ISO alto como 3200 implica que a imagem final terá muito ruído, mas possibilita que ela seja registrada com um baixo tempo de exposição. \cite{book:bbcsky} O ruído agregado pelo ISO também acaba prejudicando a fotografia reduzindo o contraste e saturação das imagens, o que também pode levar a posterização, implicando no comprometimento total da fotografia pois a foto perde resolução e criam-se falhas nos pixeis da imagem.


\begin{figure}[h]
	\centering
	\caption{Variações do ISO e o ruído agregado}
	\includegraphics[width=0.7\linewidth]{figuras/ISO}
	\label{fig:iso}
	\fonte{Adaptado de \cite{site:eduardoemonica}}
\end{figure}

\subsection{Formatos de Arquivos}

As câmeras profissionais possibilitam salvar as imagens em diferentes formatos de arquivos, que inclui formato RAW, JPG e Ambos os formatos. Os arquivos JPG são uma versão reduzida dos formatos RAW, onde se aplica um algoritmo de compressão de imagens que acaba gerando perca de informações. Desse modo, arquivos RAW possuem a informação pura do sensor, sem nenhum tipo de compactação e acabam sendo muito pesados porém permitem uma pós produção mais precisa que acaba finalizando em uma imagem com mais qualidade e detalhes. \cite{book:bbcsky}

\subsection{Rastro de Estrelas}

O movimento de rotação da terra gera um movimento aparente no céu. Ao realizar uma fotografia de longa exposição, esse movimento será visível criando o efeito de Rastro de Estrelas ou \textit{star-trail}. (Figura \ref{fig:startrail_example})

\begin{figure}[htb]
	\centering
	\caption{Fotografia com a captura de um \textit{Star Trail} (Rastro de Estrelas)}
	\includegraphics[width=0.7\linewidth]{figuras/startrail_example}
	\label{fig:startrail_example}
	\fonte{Autor}
\end{figure}

\subsection{Empilhamento de Fotos}

O fenômeno de Star Trail gera a necessidade do uso de ferramentas para compensar o movimento da terra e permitir uma fotografia de longa exposição sem que se crie rastro. Essa compensação pode ser feita por meio de software, realizando-se o empilhamentos de fotos de curta exposição. \cite{livro:astropratica}

O empilhamento consiste na junção de multiplas imagens capturadas com a câmera montada em um tripé ou em uma montagem motorizada, que possibilita o somatório da luz capturada com essas fotos. Esse método de processamento é relevante para qualquer tipo de astrofotografia e geralmente possibilita também a redução de ruído usando imagens de calibração.\cite{book:bbcsky} Existem inúmeros programas capazes de realizar esse processo como Deep Sky Stacker, Sequator entre outros.

\subsubsection{Imagens de Calibração}

\paragraph{\textit{Dark Frames}}
\paragraph{\textit{Flat Frames}}
\paragraph{\textit{Bias Frames}}
\paragraph{\textit{Dark Flat Frames}}



\subsection{Métodos de Rastreamento}


\subsubsection{Alt-Azimutal}
\subsubsection{Equatorial}

\section{Plataformas Equatoriais}

\subsection{Métodos de Alinhamento Polar}

\subsubsection{Ajuste de Azimute}

\paragraph{Localização da Estrela Polar}
Uso de lunetas para alinhamento

\paragraph{Alinhamento com o Polo Norte}
Declinação Magnética

\subsubsection{Ajuste de Elevação}

\subsubsection{Método \textit{Drift}}

\subsection{Soluções Comerciais Existentes}

Existem inúmeras soluções comerciais para o problema proposto, porém todos eles usam a luneta como método de alinhamento polar e isso só é praticável no hemisfério norte devido ao forte brilho da estrela polar diferentemente do situação no hemisfério sul. Existem produtos para todo o tamanho de orçamento. A tabela \ref{tabela_benchmark} ilustra a concorrência dos principais equipamentos, comparando as principais funcionalidades. 

\begin{table}[htb]
	\caption{Comparativo das Soluções de Mercado}
	\begin{tabular}{l|cccc}
		& Nyx Tracker & iOptron & Vixen Optics & SkyWatcher \\ \hline
		Preço (US\$) & 115 & 299 & 399 & 299 \\\hline
		Carga Máxima (kg) & 2.25 & 3 & 2 & 3 \\\hline
		Erro periódico (arcsec) & 115 & 100 & 50 & 50 \\\hline
		Volume $ (cm^2) $ & 155 & 490 & 323 & 220 \\\hline
		Peso (kg) & 0,4 & 1,15 & 0,79 & 0,72 \\\hline
		Alinhamento & \textit{Laser} & \textit{Polar Scope} & \textit{Polar Scope} & \textit{Polar Scope} \\
	\end{tabular}
	\label{tabela_benchmark}
	\fonte{\cite{site:nyxtech}}
\end{table}

Contudo, na realidade brasileira, o preço mostrado passaria ainda por impostos, tornando a compra mais inviável. O Nyx Tracker (Figura \ref{fig:nyxtracker}) é o sistema mais barato, com alinhamento impraticável no hemisfério sul, e também o mais simples em materiais.

\begin{figure}[h]
	\centering
	\caption{Nyx Tracker}
	\includegraphics[width=0.3\linewidth]{figuras/nyxtracker}
	\label{fig:nyxtracker}
	\fonte{\cite{site:nyxtech}}
\end{figure}


\section{Objetivos}

Pelo \textit{benchmark} exposto, fixou-se como objetivos do sistema um produto robusto, visualmente elegante, e que consiga se aproximar das propriedades do modelo comercial mais barato, com erro periódico igual ou inferior a 115 arcsec, peso e volume adequados para o tripé usado, mantendo-se estável, sendo o menor possível e com o preço inferior a 115 dólares. Com o diferencial de um aplicativo que permita uma fácil interação do usuário com o sistema, descomplicando o processo.

\section{Protocolos de Comunicação}

\subsection{Serial}
\subsubsection{UART}
Velocidade, Falhas de comunicação, Guidelines de Design de PCB

\subsubsection{I2C}

Endereçamento, Velocidade, Guidelines de Design de PCB

\subsection{Bluetooth}

\section{Sensores e Atuadores}

\subsection{Acelerômetro}
\subsection{Giroscópio}
\subsection{Magnetômetro}

\subsection{GPS}

\subsection{Motor de Passo}

Driver, formas de Acionamento...

\section{Microcontroladores}

\subsection{Arduino Nano}

Justificativa, diagrama do Arduíno

\section{Interface Gráfica}

\subsection{Princípios e Diretrizes}

Os princípios e as diretrizes comumente utilizados em IHC giram em torno dos seguintes tópicos: correspondência com as expectativas dos usuários; simplicidade nas estruturas das tarefas; equilíbrio entre controle e liberdade do usuário; consistência e padronização; promoção da eficiência do usuário; antecipação das necessidades do usuário; visibilidade e reconhecimento; conteúdo relevante e expressão adequada; e projeto para erros.  \cite{BarbosaEtAl2021InteracaoHumanoComputadorExperiencia}
Esse conjunto de princípios são conhecidos como heurística de Nielsen, pois são aplicáveis em qualquer sistema, independente de casos específicos.

\subsubsection{Visibilidade dos status do sistema}

O sistema deve sempre manter o usuário atualizado sobre as condições de operação com uma taxa de atualização condizente para a informação. Ao informar o status da bateria, por exemplo, o usuário do smartphone consegue predizer quanto tempo de uso ainda terá e irá conseguir manejar sua interação com base nessa previsibilidade. \cite{site:nielsen}

\subsubsection{Comunicar-se com o mundo real}
O Design tem que se comunicar com o usuário na língua do usuário. Se um brasileiro não sabe inglês, ele ficará perdido nos Estados Unidos, pela mesma lógica, se a máquina não consegue se comunicar com o usuário, então ele ficará perdido.

Da mesma forma, o desenvolvedor não pode assumir que o usuário entenderá o aplicativo somente pelo fato do desenvolvedor ter feito algo que ele próprio entenda; é sempre preciso conferir a linguagem do sistema com um conjunto grande de pessoas para evitar mal entendidos, que se já ocorrem em língua nativa, na língua do sistema só tende a piorar.

Quando o usuário não entende a língua do sistema, ele se sente afastado e irá deixar de usar a plataforma. É interessante que a plataforma tenha designs semelhantes com objetos do mundo real, dessa forma, o usuário se sente "contemplado" e consegue facilmente fazer a conexão entre o mundo real e a plataforma. \cite{site:nielsenRealWorld}

\subsubsection{Liberdade de Controle do Usuário}

Por vezes, a pessoa que está realizando um processo em um sistema pode cometer um engano. Esse evento pode levar a situações de erro que não devem comprometer a experiência. Por isso, os usuários precisam de uma “saída de emergência” claramente marcada para sair do estado indesejado. Isso reduz a sua ansiedade e o medo de errar, pois ele sabe que os erros podem ser desfeitos. \cite{BarbosaEtAl2021InteracaoHumanoComputadorExperiencia}

\subsubsection{Consistências e Padrões}

É importante que o sistema mantenha uma consistência entre suas telas, ou mesmo em grandes plataformas, que os múltiplos programas tenham a mesma 'cara' com funções localizadas no mesmo lugar, com nomes similares e com um disign similar. Exeplo disso é as telas dos aplicativos do google docs; todos possuem o mesmo extilo de menu. MS Office também oferece isso.

A consistência também se extende aos ícones. O vetor que representa um botão, por exemplo, é importante que ele seja consistente em extilo com os demais. Eles podem ser mais preenchidos, mais \textit{clean}, mais neturos, mais suaves. O que importa mais nesse caso é que sejam todos padronizados. \cite{site:nielsenIcon}


\subsubsection{Prevenção de erros}

Uma forma de prevenção é oferecer sugestões numa caixa de pesquisa por exemplo. Em situações de rotina, como disparar um lembrete, a tela de criação pode oferecer uma sugestão defaut de template que faça sentido de fato para o usuário. Para evitar corrupção de dados pelo usuário na hora de um cadastro, é possível ofercer ao usuário que ele preencha números de uma forma truncada, e fazendo um pós processamento para ler o número corretamente.

\subsubsection{Relembrar o usuário é mais fácil do que o usuário relembrar}

Quando o usuário precisa repensar sobre algo incomoum na memória, ele precisa de muito tempo para pensar sobre esse algo. Então, quando a plataforma exige uma lembrança do usuário para entender algo, acaba que isso limita a experiência e incorre em perda de tempo, ou confusão.

Por isso, é mais interessante realizar a exigência com uma possível sugestão de resposta correta. A recognição de algo é muito mais prática para a mente humana pois ao mostrar para o cérebro algo relacionado com o que precisa lembrar-se, isso dispara a memória de forma mais forte. Dar uma pista para o cérebro é mais eficiente do que simplesmente perguntar sem oferecer nada para a memória. \cite{site:nielsenRecall}

\subsubsection{Torne o sistema flexível e eficiente}

Atalhos, personalização e customização. Com esses 3 fatores é possível melhorar a usabilidade para aqueles que não são mais novatos no software e isso ajuda a manter esses usuários ativos. Um fotógrafo experiente, que está acostumado com os atalhos de teclado nos aplicativos da Adobe, teria muita "dor de cabeça" se o teclado viesse a falhar, pois a mente já assimilou os atalhos mais usados e eles fazem diferença na velocidade com que o profissional atua com o software.
\cite{site:nielsenFlexibility}

\subsubsection{Tenha um Design minimalista}

Um design minimalista ajuda a ter somente o que é necessário focar na tela, isso ajuda o usuário a não se sentir perdido. Isso significa usar elementos simples num arranjo onde desenho e a interface combine de forma agradável sem chamar a atenção de forma desnecessária. \cite{site:nielsen}

\subsubsection{Ajude o usuário a entender e se recuperar de erros}

O usuário precisa entender quando o sistema não está funcionando bem e como fazê-lo voltar ao normal. As mensagens de erro devem ser expressas em de uma forma simples, indicando o possível problema e a solução. 
Cores vermelhas e pretas ajudam a demonstrar o sinal de erro para o usuário. \cite{site:nielsenError}

\subsubsection{Tire dúvidas e documente o sistema}

Existem duas formas de ajudar o usuário e tirar suas dúvidas. A primeira é de forma proativa, onde a aplicação guia o usuário para se familiarizar com a interface. Outra forma é por meio de uma seção com perguntas e respotas, essa seção ajuda os usuários a se tornarem mais independetes com a aplicação, resolvendo seus próprios problemas e filtrando os casos que precisam de suporte para a equipe técnica da plataforma. \cite{site:nielsenHelpandDoc}

\subsection{Android}
\subsubsection{Ambiente de Desenvolvimento}

O \textit{Android Studio} é o ambiente de desenvolvimento integrado oficial para a criação de apps \textit{Android} e é baseado no \textit{IntelliJ IDEA}. Ele oferece uma série de Recursos que possibilitam a confecção de um aplicativo: Sistema de compilação flexível baseado em \textit{Gradle}; Um emulador rápido com suporte a vários recursos; um ambiente unificado que possibilita o desenvolvimento para qualquer dispositivos Android, incluindo relógios e televisões; Integração com GitHub para backup e documentação do código; entre outras funções que possibilitam análizar o desempenho de um aplicativo em tempo real, bem como fazer updates. \cite{site:androidstudio}

\subsubsection{Linguagens de Programação}

Existem soluções de desenvolvimento Android mais \textit{user-friendly} como \textit{APP Inventor} ou \textit{Kodular}, porém, essas interfaces não garantem ao desenvolvedor um pleno controle do aplicativo, e muitas vezes acaba limitando a interface. Por isso, usar linguagens de programação nativas é uma abordagem mais interessante para aplicativos mais completos. É possível criar aplicativos com diversas linguagens, mas somente duas são nativas e permitem realizar aplicações que podem abusar de todo o poder de processamento de um \textit{smartphone}: Java e Kotlin.

Em 2017, Kotlin foi definido pela Google como sendo a principal linguagem de desenvolvimento Android. A linguagem ainda é muito mais nova que Java, sendo desenvolvida em pela JetBrains. As grandes motivação de se usar Kotlin para o desenvolvimento reside nos fatos de ser uma linguagem segura para prevenção de objetos nulos, operando em paralelo com qualquer código em Java e dando opções de co-rotinas. Além disso, ao comparar dois códigos com a mesma função, um escrito em Java, outro em Kotlin; o segundo pode ser até 40\% mais compacto, o que implica em uma linguagem mais concisa e compreensível entre desenvolvedores. As desvantagens de se usar Kotlin, para este trabalho, é somente a falta de uma comunidade grande, comparando com Java; o que limita o suporte para eventuais problemáticas de desenvolvimento. \cite{site:kotlinxjava}

Dentro do ambiente de Desenvolvimento usa-se também linguagem de arquivos XML para a criação de interfaces gráficas (\textit{layouts}) bem como a escrita dos vetores, animações, e arquivos de configuração do aplicativo e temas de \textit{layout}. Para armazenamento de dados dentro do aplicativo normalmente usa-se uma \textit{database} que é operada com códigos de consulta \textit{SQL}.

\subsubsection{\textit{Material Design}}
Existem uma série de diretrizes de Design fornecidas pela Google para guiar o desenvolvimento de aplicativos android. Essas informações são fornecidas principalmente pela biblioteca Material Design, que fornece pacotes facilmente implementáveis de layouts para aplicações responsivas e padronizadas.

A Biblioteca colabora com o desenvolvedor fornecendo icones, tipografia, cores e componentes gráficos que trazem uma imersão para o usuário de forma simples e minimalista. Os design se inspiram no mundo real, facilitando a comunicação com o usuário. \cite{site:materialdesign}



