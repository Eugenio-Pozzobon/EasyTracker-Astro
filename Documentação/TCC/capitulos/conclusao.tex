\chapter{Considerações Finais}

Os objetivos do trabalho foram elaborados com base em \textit{benchmark} comercial, e foi possível vencer as expectativas criando um protótipo economicamente mais viável, como massa total dentro da média comercial, que ao mesmo tempo é mais \textit{user-friendly} com o público-alvo. Dentre os sub-objetivos de projeto, o erro do motor é melhor do que dispositivos comerciais; e os limites de peso suportado também se encontram dentro da margem esperada para o porte do protótipo. A Tabela \ref{tabela_benchmark_conclusao} sintetiza a comparação do sistema com modelos comerciais, demonstrando os resultados atingidos.


\begin{table}[htb]
	\caption{Comparativo das Soluções de Mercado}
	\begin{tabular}{l|cccc}
		& EasyTracker & Nyx Tracker & iOptron  & SkyWatcher \\ \hline
		Preço (US\$) & 62\footnote{Considerando taxa de câmbio para o dia 5 de maio de 2022, quando 1 dólar vale 4,94 reais. } & 115 & 299  & 299 \\\hline
		Carga Máxima (kg) & 2,5 & 2,25 & 3 & 3 \\\hline
		Erro periódico (arcsec) & 63 & 115 & 100 & 50 \\\hline
		Peso (kg) & 0,716 & 0,4 & 1,15 & 0,72 \\\hline
		Alinhamento & Sensores e App. & \textit{Laser} & \textit{Polar Scope} & \textit{Polar Scope} \\
	\end{tabular}
	\label{tabela_benchmark_conclusao}
	\fonte{Adaptado de \cite{site:nyxtech}.}
\end{table}

O aplicativo, por outro lado, é o grande diferencial; com ele foi possível simplificar todo o processo de alinhamento; e os \textit{feedbacks} do usuário convidado demonstram sua eficiência em realizar um alinhamento bom com um curto período de tempo. Essa abordagem inovadora também foi motivo de atenção na mídia, a Revista Arco da UFSM\footnote{Link da reportagem na revista Arco: \url{https://www.ufsm.br/midias/arco/a-arte-de-fotografar-o-espaco/}}, o G1\footnote{Link da reportagem no G1: \url{https://g1.globo.com/rs/rio-grande-do-sul/noticia/2021/12/30/pesquisadores-da-ufsm-criam-dispositivo-que-ajuda-fotografos-amadores-na-pratica-da-astrofotografia.ghtml}} e a RBS TV\footnote{Link da reportagem no Jornal do Almoço da RBS TV: \url{https://globoplay.globo.com/v/10171682/}} divulgaram o trabalho em suas plataformas.


Com isso o projeto passou a ser alvo de muitos interessados em obter um protótipo. No entanto, como não é necessariamente uma obrigação ou desejo dos autores fazer do EasyTracker um produto comercial, e pensando no futuro do desenvolvimento deste, optou-se por manter o projeto como \textit{Open-Source}.

\section{Projeto \textit{Open-Source}}

Então, o EasyTracker foi integralmente disponibilizado no GitHub na esperança de que ele possa ser reproduzido por quem interessar, e seja melhorado pela comunidade. Durante as etapas em que houve contato com o público, foi possível perceber que existem diversos outros usos para o conceito que foi proposto. E, com isso, a melhor forma de fazer com que essas variações do trabalho realmente venham a florescer, é tornando o projeto público, acessível e bem documentado.

Contudo, apesar de livre, ressalta-se que o projeto de software embarcado e Android estão licenciados sob proteção da GPLV3. A licença foi aplicada em cada arquivo de software, como consta em suas diretrizes. O Trabalho, portanto, não pode ser reproduzido sem que os autores sejam mencionados. Além disso, tudo aquilo que for produzido com base neste material, também deve ser disponibilizado de forma \textit{open-source}, e com a mesma licença GPL. 

A documentação desse projeto, que muitas vezes foi referenciada neste documento em notas de rodapé, está protegida sob a licença \textit{Creative Commons (CC BY-SA)}. Com ela, essa documentação pode ser redistribuída e alterada, desde que os autores recebam os devidos créditos \cite{CCBYSA}. Todos os arquivos licenciados carregam a marca dessa licença. 


Por fim, o projeto eletromecânico está apenas disponível de forma online e sem licença. As leis internacionais de patente referente a componentes mecânicos impedem que este projeto possa ser patenteado, uma vez que já está disponível na internet para amplo acesso. Mesmo que fosse do interesse realizar uma patente, a parte mecânica não seria permitida pois os projetos e modelos de rastreadores \textit{Barn-Door} apresentados, já estão publicados desde a década de 80.


\section{Sugestões para o futuro do EasyTracker}

Finalmente, o projeto não termina com a elaboração deste documento. Existem muitos diferenciais na parte eletrônica e de \textit{software}, que podem ser expandidos para outros projetos e utilidades. Por exemplo, é possível reaproveitar o projeto, remodelando-o para ser usado em conjunto com sistemas mecânicos profissionais. Nesse cenário, o projeto passa a ser um "extra" para aqueles que já possuem um sistema de rastreamento importado. 

Contudo, para aqueles que desejam desenvolver e montar o seu próprio EasyTracker, melhorias podem ser implementadas, ou mesmo customizações podem ser criadas. O inserto da base inferior, usado para fixação no tripé, por exemplo, pode ter variações dependendo do padrão empregado no tripé que a pessoa utiliza, assim como o \textit{ball-head} de fixação da câmera. Recomenda-se ainda utilizar um padrão 3/8" uma vez que este fixador é mais resistente que o 1/4", e consegue prover uma melhor fixação à plataforma.

Para continuações do projeto, é interessante uma evolução do código no Arduíno e do Aplicativo. A proposta de apoio na interface para realização do método \textit{Drift} -- discutida e projetada, respectivamente, nas seções \ref{driftsection} e \ref{sec:driftproj} -- ainda pode ser implementada no aplicativo. Para a eletrônica, recomenda-se a substituição do Arduíno por um ESP32 com Bluetooth integrado.

O ESP32 é um microcontrolador voltado para sistemas IOT (\textit{Internet of Things}), capaz de realizar \textit{multithread} com seu \textit{chip dual-core} \cite{man:ESP32}. Assim, é possível aplicar uma \textit{thread} para leitura do sensor com comunicação \textit{bluetooth}, e outra para o controle do motor de passo; ambas são funções que não podem funcionar totalmente em simultâneo com um Arduíno.

Com essa alternativa de controlador, é possível simplificar a placa, tendo apenas o ESP, alimentação e o sensor. O conjunto de instruções programados em C/C++ pode ser reaproveitado pois ele também é programado com a mesma linguagem de programação.