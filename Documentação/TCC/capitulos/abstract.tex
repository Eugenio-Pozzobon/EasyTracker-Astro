Currently, astrophotography remains practiced by large telescopes and also by astrophotographers that have countless challenges. The main one relies on the fact that celestial bodies, in general, demand long exposure times. Unfortunately, despite the camera being on a tripod, the rotational movement of the Earth doesn't allow the stars to be exposed to the sensor for a long time, resulting in blurry images. For this reason, it is necessary to use a tool that moves the camera in the apparent rotation direction of the sky, compensating for this movement and obtaining a high-quality photographic record. For this, there are numerous commercial tools for tracking the sky. However, all of them are commercialized in the Northern hemisphere and with an exorbitant cost for the average Brazilian. So, in order to simplify and reduce the cost associated with astrophotography, making it accessible, the objective of this work was to develop an equatorial platform for astrophotography that is portable; robust; precise; easy to set up and use; of weight and volume compatible with photographic tripods; and at a lower cost than commercial solutions. The platform's differential is a mobile application that makes it easy to use these tools for the configuration of the platform to obtain photographic records. The final system has been passed into vibration tests to ensure that the assembly would be suitable for use.  Also, the system got tested with long exposure photography.