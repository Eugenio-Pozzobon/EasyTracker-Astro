Atualmente, a astrofotografia é realizada por telescópios monumentais e também por um grande número de astrofotógrafos amadores, e estes contam com inúmeros desafios. O principal deles reside no fato que corpos celestes, de forma geral, demandam elevados tempos de exposição. Infelizmente, caso a câmera esteja imóvel sobre um tripé, o movimento de rotação da Terra não permite que os astros sejam expostos ao sensor por muito tempo, pois as imagens acabam borradas. Por esse motivo, é necessário o uso de uma ferramenta que movimente a câmera no sentido de rotação aparente do céu, compensando esse movimento e obtendo-se um registro fotográfico de alta qualidade. Para isso, existem inúmeras ferramentas comerciais para o rastreamento do céu, porém, todas elas são comercializadas no hemisfério Norte e com um custo excessivo para o brasileiro médio. Então, a fim de simplificar e reduzir o custo associado à astrofotografia, tornando-a acessível, objetivou-se com este trabalho desenvolver uma plataforma equatorial para astrofotografia que seja portátil; robusta; precisa; de fácil configuração e utilização; de peso e volume compatíveis com tripés fotográficos; e com custo inferior à soluções comerciais. A plataforma tem como diferencial um aplicativo mobile que descomplica o uso dessas ferramentas para configuração e setup da plataforma para a obtenção de registros fotográficos. O sistema final passou por testes de vibração para garantir que a montagem estaria adequada para o uso, e também com testes em campo, onde foram tiradas fotografias da via-láctea, comparando-se resultados fotográficos com e sem o uso da plataforma desenvolvida.

