Atualmente, a astrofotografia é realizada por grandes telescópios e também por um crescente número de astrofotógrafos amadores, os quais contam com diversas limitações. A principal reside no fato de que objetos celestes, de forma geral, demandam elevados tempos de exposição. Infelizmente, caso a câmera esteja imóvel sobre um tripé, o movimento de rotação da Terra não permite que os astros sejam expostos ao sensor por muito tempo, pois as imagens acabam borradas. Por esse motivo, é necessário o uso de uma ferramenta que movimente a câmera no sentido de rotação aparente do céu, compensando esse movimento e permitindo um registro fotográfico de alta qualidade. Para isso, existem inúmeras ferramentas comerciais para o rastreamento do céu, porém, todas elas são comercializadas no hemisfério Norte e com um custo expressivo para o brasileiro médio. Então, a fim de simplificar e reduzir o custo associado à astrofotografia, tornando-a mais acessível, objetiva-se neste trabalho desenvolver uma plataforma equatorial para astrofotografia que seja portátil, robusta, precisa, de fácil configuração e utilização, de peso e volume compatíveis com tripés fotográficos, e com custo inferior à soluções comerciais existentes. A plataforma tem como diferencial um aplicativo \textit{mobile} que auxilia a configuração da plataforma para a obtenção de registros fotográficos. O sistema final passou por testes de rastreamento (erro de velocidade) e de vibração para garantir que a montagem estaria adequada para o uso, e também com testes em campo, onde foram obtidas fotografias da Via-Láctea.

